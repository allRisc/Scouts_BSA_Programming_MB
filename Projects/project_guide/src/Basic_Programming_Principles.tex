\chapter{Basic Programming Principles}
\label{app:programming_basics}

  This appendix is intended to provide some background information about how programming generally works.
  The examples here are for reference as many of these principles vary in how they are coded depending on the language.
  If you are working on a specific programming language \texttt{Google} or some other search engine can certainly help.

  \section{Comments}
  \label{sec:basics_comments}

    Comments are bits of text in code which are ignored when the code is run.
    But that doesn't mean that comments aren't important!
    In fact comments are some of the most useful parts of programming.
    Comments allow you make observations (or comments) about how your code works inside the code itself.
    This is useful since it allows you to ensure you don't forget this information when you leave the code and return to it later.

    \begin{lstlisting}[
      language=Python,
      backgroundcolor=\color{backcolor},
      commentstyle=\color{codegreen}
    ]
      # This is a comment in Python
    \end{lstlisting}

    \begin{lstlisting}[
      language=Java,
      backgroundcolor=\color{backcolor},
      commentstyle=\color{codegreen}
    ]
      // This is a comment C, C++, JavaScript, Java, and others

      /* You 
         can
         also 
         write
         multi-line 
         comments
         like
         this */
    \end{lstlisting}